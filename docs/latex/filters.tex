
\documentclass{report}
\begin{document}
\chapter{Genotype Filters}
Some of the detected trajectories or genotypes do not represent real mutations, and need to be removed from the analysis. 

Common situations where a trajectory or genotype should be removed:

\begin{enumerate}
\item A mutation appears before and after a genotype sweeps and removes all pre-existing variation. 
        Since the mutation should have been removed when the genotype fixed
\item A mutation fixes immediately during the same timepoint it is detected (there are no intermediate values), then becomes undetected again.
        This is likely due to a sampling error.
\item Similar to above, a mutation is detected at a single timepoint. 
        These mutation do not provide much information and complicate the analysis, so they are filtered out.
        \verb|--disable-filter-single| disables both this and the above filter.
\item The first timepoint shows that a mutation has already fixed. 
        This contradicts the known biology of the experiment.
        \verb|--disable-filter-startsfixed| disables this filter.
\item A mutation is relatively constant throughout the experiment and is not affected by the movement of other mutations.
        This filter can be disabled by setting \verb|--filter-constant| to 0.
\end{enumerate}

An exception has been made for the case where a mutation appears both before and after a genotype sweeps, but is undetected at the timepoint where this occurs. 
This represents a mutation arising, being removed during a genotype sweep, then arising again.
Note that when a genotype is removed, all of the trajectories that comprise that genotype are removed and the clustering algorithm is re-run on the remaining dataset. 
\end{document}