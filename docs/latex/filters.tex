\documentclass{report}
\begin{document}
\chapter{Genotype Filters}
Some of the detected trajectories or genotypes do not represent real mutations, and need to be removed from the analysis. 
Since all pre-existing variation is removed when a specific genotype sweeps and fixes, any mutation that is seen before, during, 
and after the timepoint where the sweep occured are likely due to measurement error.

Common situations where a trajectory or genotype should be removed:

\begin{enumerate}
\item A mutation appears before and after a genotype sweeps and removes all pre-existing variation. Since the mutation should have been removed when the genotype fixed
\item A mutation fixes immediately during the same timepoint it is detected (there are no intermediate values), then becomes undetected again.
\end{enumerate}

An exception has been made for the case where a mutation appears both before and after a genotype sweeps, but is undetected at the timepoint where this occurs. 
This represents a mutation arising, being removed during a genotype sweep, then arising again.
Note that when a genotype is removed, all of the trajectories that comprise that genotype are removed and the clustering algorithm is re-run on the remaining dataset. 

\end{document}