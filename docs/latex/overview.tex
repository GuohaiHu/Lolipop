\documentclass{report}
\begin{document}
\chapter{Overview}

\section{What is a muller diagram?}

Muller diagrams excel at visualizing evolutionary dynamics of a population over time. 
A common use case, and the intended use case during development, describes the abundance and succession of genotypes within a population derived from 
sequencing data from samples at specified timepoints. 
This provides a useful method of quickly discerning the evolutionary relationship between genotypes seen in a population and how this changes over time.


\section{Some Terminology}

\begin{description}
\item[trajectory/mutational trajectory] A set of frequency measurements of a specific mutation at each sampled time point over the course of an evolution experiment.

\item[genotype] A group of trajectories which follow a common path.

\item[clustering] The process of grouping trajectories together to form a genotype. There are two basic types of clustering. Agglomerative clustering describes the case where trajectories or small clusters are grouped together to form a larger cluster. Divisive clustering occurs when an existing cluster is split up into smaller clusters. These scripts use both types of clustering.

\item[uncertainty] A parameter provided by the user which indicates the uncertainty in the frequency measurements. The default value of 0.03 was chosen based on the performance of Breseq.

\item[background] When a mutation arises in a population, it is said to be in the background of the mutations that appeared in the same population prior to its detection. A core purpose of these scripts is to describe how any given mutation is related to those that arose before its detection (i.e. its background).

\item[genotype fixing/sweeping] When a genotype fixes, it removes all pre-existing variation other than itself. All subsequent mutations arise in the background of this genotype.
\end{description}

\end{document}