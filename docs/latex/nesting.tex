
\documentclass{book}
\usepackage{amsmath, nccmath}
\usepackage{geometry}
\newtheorem{ex}{\bf Example}[section]
% Use variables to refer to the nested/unnested genotypes in case I was to change the nomenclature later.
\newcommand{\unnestedlabel}{unnested genotype}
\newcommand{\ancestorlabel}{potential ancestor}

% Used to identify the parameters which affect this step of the workflow.
\newcommand{\parameteradditive}{additivecutoff}
\newcommand{\parametersubtractive}{subtractivecutoff}
\newcommand{\parametercovariance}{covariancecutoff}
\begin{document}
\chapter{Nesting successive genotypes}

\section{Relevant Parameters}
These parameters control how the lineage of genotypes is inferred:
\begin{description}
  \item[--additive 0.05] The \parameteradditive{} value controls when the sum of a pair of genotypes is considered consistently greater than the fixed cutoff value.
    The default value is controlled by the \verb|detection| parameter. 
  \item[--subtractive 0.05] The \parametersubtractive{} value controls when a \ancestorlabel{} is considered greater than an \unnestedlabel{} and \textit{vis versa}.
  \item[--covariance 0.01] Defines how each of the three possible covariance scenarios are defined (see below).
\end{description}  

\section{Evidence of lineage}
After grouping mutational trajectories into genotypes, the scripts infer the lineage of each genotype based on the frequency measurements.
Genotypes are nested according to a few basic rules:

\subsection{Summation}
Checks whether the \unnestedlabel{} and \ancestorlabel{} consistently sum to greater than the fixed cutoff limit (typically 0.97 to 1). 
This is weak evidence that one of the genotypes is a background and the other arises in the background of the first. 

\subsection{Genotype Size}
This checks whether one of the genotypes is consistently larger than the other. 
A genotype is considered consistently greater than another if the difference between the two frequencies exceeds a value of 0.15 
(based on original scripts) at least once or exceeds 0.03 (based on the uncertainty option) at least twice. 
\textbf{Note: This is currently combined with the subtractive check when scoring each genotype pair.}

\subsection{Covariance}
The covariance between two series of random variables provides a measure of the correlation between both series. 
A genotype which arises in the background of another genotype must be correlated with the parent genotype.
A negative covariance between two genotypes indicates that they are competing genotypes with one outcompeting the other.
The covariance of two series $X$ and $Y$ is defined as
\begin{equation}
Cov(X,Y) = \frac{1}{n}\sum_{i=0}^n (X_i-\bar{X})(Y_i-\bar{Y})
\end{equation}


\subsection{Expected vs. Observed Jaccard Distance}

The jaccard distance is a measurement of the dissimilarity of two sample sets. 
Since each frequency measurement describes the abundance of a mutation at a sampled timepoint, 
we can use the jaccard distance to compare the abundance of two genotypes over the course of an experiment.
Since each genotype represents a set of abundance measurements, the jaccard distance between two genotypes $X$ and $Y$ can be calculated as follows:
\begin{equation}
J(X,Y)=1 - \frac{|X\cup Y|}{|X|+|Y|+|X \cap Y|}\equiv \frac{|X\cup Y|-|X \cap Y|}{|X \cup Y|}
\end{equation}
Since each genotype is a set of frequency measurements, the cardinality of each set can be computed as
\begin{equation}
|X| = \sum_{i=0}^n X_i
\end{equation}


Based on this, the union and intersection can be defined as
\begin{align*}
|X \cap Y| &= \sum_{i=0}^n min(X_i,Y_i) \\
|X \cup Y| &= |X| + |Y| - |X \cap Y| \\
\end{align*}


When $Y$ arises in $X$, all elements of $Y$ are also contained in $X$, $|X \cup Y| = |X|$ and $|X \cap Y| = |Y|$, reducing the above equation to
\begin{equation}
J_{expected}(X,Y) = \frac{|X|-|Y|}{|X|}
\end{equation}


If the computed jaccard distance is equal to the expected jaccard distance, genotype $Y$ arises in genotype $X$.

\subsection{Example}

  For the genotypes listed below, how should the genotypes be nested?

\begin{tabular}{l|lllllll}
  Genotype & 0 & 17 & 25 & 44 & 66 & 75 & 90 \\
  \hline
 genotype-6 & 0 & 0  & 0  & 0.273 & 0.781 & 1.000 & 1.000 \\
 genotype-7 & 0 & 0  & 0  & 0.403 & 0.489 & 0.057 & 0.080 \\
 genotype-3 & 0 & 0  & 0  & 0     & 0.211 & 0.811 & 0.813 \\
\end{tabular}


The jaccard distance between genotype-6 and genotype-7, assuming genotype-6 is the background is calculated as follows:
\newline
\begin{align*}
 |X| &= 0.273+0.781+1.000+1.000 = 3.054 \\
 |Y| &= 0.403+0.489+0.057+0.080=1.029 \\
 |X \cap Y|&=0.273+0.489+0.057+0.080=0.899 \\
 |X \cup Y| &\equiv |X|+|Y|-|X \cap Y|= 3.054+1.029-0.899=3.184 \\
\end{align*}
\newline

Now let's calculate the jaccard distance:
\[
J(X,Y)=\frac{|X\cup Y|-|X \cap Y|}{|X \cup Y|}=\frac{3.184-0.899}{3.184}=0.718
\]

Test against the expected distance, computed as:
\[
J_{expected}(X,Y) = \frac{|X|-|Y|}{|X|}=\frac{3.054-1.029}{3.054}=0.663
\]

Since $0.718\ne 0.663$, We cannot say that $Y$ arises in the background of $X$.

\subsection{Example}
Let's test if genotype-3 arises in the background of genotype-6:

\begin{align*}
|X| &= 0.273+0.781+1.000+1.000 = 3.054 \\
|Y| &= 0.211+0.811+0.813=1.835 \\
|X \cap Y|&=0.000+0.211+0.811+0.813=1.835 \\
|X \cup Y| &\equiv |X|+|Y|-|X \cap Y|= 3.054+1.835-1.835=3.184 \\
\end{align*}
The jaccard distance is then
\[
J(X,Y)=\frac{|X\cup Y|-|X \cap Y|}{|X \cup Y|}=\frac{3.184-1.835}{3.184}=0.425
\]
Test against the ideal distance:
\[
J_{expected}(X,Y) = \frac{|X|-|Y|}{|X|}=\frac{3.054-1.835}{3.054}=0.425
\]

Since $J = J_{expected}$ there is evidence that genotype-3 arises in the background of genotype-6.


\section{Scoring}
Each of the above checks is used to calculate a score between each \unnestedlabel{} and a \ancestorlabel{}. 
The parent genotype is then chosen as the potential ancestor with the greatest score when compared against the given genotype.

\begin{description}
  \item[additive score] \textbf{[0,1]} Tests whether the combined \unnestedlabel{} and \ancestorlabel{} consistenty sum to greater than $1$. 
  The only scenario where this should occur is if the \unnestedlabel{} arises in the background of \ancestorlabel{}.
  \begin{description}
    \item[1] The average sum of the \unnestedlabel{} and the \ancestorlabel{} is greater than 1.
    \item[0] The average sum of the \unnestedlabel{} and the \ancestorlabel{} is equal or below 1. 
  \end{description}
  \item[subtractive score] \textbf{[-1,1]} The subtractive score adds weak evidence for/against the hypothesis that the \ancestorlabel{} is a true ancestor of the \unnestedlabel{}.
    \begin{description}
      \item[1]  The \unnestedlabel{} is always detected at a lower frequency than the \ancestorlabel{}.
      \item[0]  The \unnestedlabel{} is occasionally detected at a greater frequency than the \ancestorlabel{}.
      \item[-1] The \unnestedlabel{} is always detected at a greater frequency than the \ancestorlabel{}.
    \end{description}
  \item[jaccard score] \textbf{[-2,2]} The \unnestedlabel{} is compared against both the \ancestorlabel{} and a hypothetical genotype consisting of the remaining
    population not represented by the \ancestorlabel{}. This hypothetical genotype is assigned a frequency of $0$ for all timepoints where \ancestorlabel{} is 
    fixed and a frequency of $1-f_i$ for all timepoints that the \ancestorlabel{} is not fixed. 
    \begin{description}
      \item[2]  The \unnestedlabel{} is only a subset of the \ancestorlabel{}.
      \item[1]  The \unnestedlabel{} is a subset of both the \ancestorlabel{} and the hypothetical genotype, but overlaps with the \ancestorlabel{} twice as much
        as the hypothetical genotype.
      \item[0]  The \unnestedlabel{} is a subset of both the \ancestorlabel{} and the hypothetical genotype, 
        but shares a similar or smaller overlap with the \ancestorlabel{} as it does with the hypothetical genotype. 
      \item[-2] The \unnestedlabel{} is a subset of only the hypothetical genotype, and is not a subset of the \ancestorlabel{}
    \end{description}  

  \item[covariance score]  \textbf{[-2,2]} Tests whether the two genotypes are competing against each other or mutually evolving together.
    The covariance $Cov(X,Y)$ between the two genotypes is compared against a cutoff value $d$ (see the program options for more information) to test which 
    scenario this pair of genotypes falls under. 
    \textbf{Note: This is only checked if the sum of the previous scores is positive.}
    \begin{description}
      \item[2]  $Cov(X,Y) > d$
      \item[0]  $-d \le Cov(X,Y) \le d$
      \item[-2] $Cov(X,Y) < -d$ 
    \end{description} 
\end{description}

\subsection{Example}

\end{document}
